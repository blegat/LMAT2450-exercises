\newape
\begin{solution}
  Let assume that f is not one way function $y = f(x)$.
  Then, we can recover $x$ with a non negligible probability $\epsilon_x (n)$.
  
  So, $(y_0, y_1) \Rightarrow (x_0, x_1)$ with probabilities $(\epsilon_{x_0} (n), \epsilon_{x_1} (n))$.
  We cannot compute a pre-image by asking the oacle I. So I output $(0, x_1)$ and $(1, x_2)$ as a forgery.

  The probability $Pr[Forge(y)] = 2\epsilon(n)$, non-negiligble since f is not one way.
\end{solution}
\begin{solution}
  \begin{enumerate}
    \item
      A has $(i, \sigma(i))$ whith $\sigma (i) = f^{n-i} (x)$. We know (because $f$ is a permutation function) that:
      $$f(\sigma(i)) = f^{n-i+1}(x) = f^{n-(i-1)}(x) = \sigma(i-1)(x)$$
      Then it's possible to compute a valid forgery for every $j < i$. The scheme is then not a one time-signature.
    \item
      Need schema drawn at TP !! It's a lot simpler with it...

      $Pr[Success (A_\sigma)] = \epsilon(\lambda))$, $Pr[Abort] = \frac{n-k}{n}$, $Pr[Success] = \frac{n-k}{n-m-1}$ then:
      $$ Pr[Success(A_{owf})] = \epsilon(\lambda) \frac{n-k}{n} \frac{n-k}{n-m-1}$$

      If $\epsilon(\lambda)$ is not negligible, then the probability of success is not negligible.
    \item
      We have $s_k = (x, x')$, $p_k = (f^n(x), f^n(x'))$.
      Then $m \rightarrow \sigma = (f^{n-m}(x), f^m(x'))$.
      
  \end{enumerate}
\end{solution}
\begin{solution}
  \begin{enumerate}
    \item
      $(r, s) = ([g^k \pmod{p}] \pmod{q}, [H(m) + xr]k^{-1} \pmod{q})$, then:
      $u_1 = H(m)s^{-1} \pmod{q}$,  $u_2 = rs^{-1} \pmod{q}$, $r = [g^{u_1} g^{u_2} \pmod{p}] \pmod{q}$, $y = g^x$
      
      $$\Rightarrow g^{u_1 + xu_2} = g^{s^{-1}(H(m) + rx)} = [g^k \pmod{p}] \pmod{q} = r$$
    \item
      $s = (H(m) + xr)k^{-1} \pmod{q}$, $s' = (H(m') + xr)k^{-1} \pmod{q}$ ($s \neq s'$ otherwise we have a collision).
      $s - s' = (H(m) - H(m'))k^{-1} \pmod{q}$, $k = \frac{H(m) - H(m')}{s - s'}$.
      $s = (H(m) + xr)k^{-1}$ so $\frac{sk - H(m)}{r} = x$ where x is the secret.
  \end{enumerate}
\end{solution}
\begin{solution}
ape7

\end{solution}


% Next are copies of ape 4
\begin{solution}
See APE4, exercice 1
\end{solution}
\begin{solution}
See APE4, exercice 2
\end{solution}
\begin{solution}
See APE4, exercice 3
\end{solution}
\begin{solution}
See APE4, exercice 4
\end{solution}
\begin{solution}
See APE4, exercice 5
\end{solution}
